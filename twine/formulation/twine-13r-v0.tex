% generated by plotdistinguisher.py
\documentclass[preview]{standalone}
\usepackage{comment}
\usepackage{tugcolors}
\usepackage{twine}
\usepackage[margin=2.65in]{geometry}
%\tikzset{warpfig/.append style={white}}
\begin{document}
\begin{figure}[htp!]
\begin{tikzpicture}[twinefig]
\foreach \z in {32,33,34,35,36,37,38,39,48,49,50,51,52,53,54,55,60,61,62,63} { \fill[tugred] (.25*\z,.75) circle[radius=3pt]; }
   \foreach \z[evaluate=\z as \zf using int(4*\z)] in {0,...,15} {
       \draw[gray] (\z,0) node[above] {\tiny\zf};
       \foreach \zb in {0,...,3} { \draw[gray] (\z+.25*\zb,0) -- +(0,-3pt); }
}
\end{tikzpicture}
\twineroundwokey{%s
\markevenbranches{8/13,12/15}{markupperpath}{->}
\markoddbranchbeforexor{9,13,15}{markupperpath}
\markoddbranchafterxor{15/14}{markupperpath}{->}
}
\twineroundwokey{%s
\markevenbranches{14/11}{markupperpath}{->}
\markoddbranchbeforexor{13,15}{markupperpath}
\markoddbranchafterxor{13/10}{markupperpath}{->}
}
\twineroundwokey{%s
\markevenbranches{10/9}{markupperpath}{->}
\markoddbranchbeforexor{11}{markupperpath}
\markoddbranchafterxor{}{markupperpath}{->}
}
\twineroundwokey{%s
\markevenbranches{}{markmidupperpath}{->}
\markoddbranchbeforexor{9}{markmidupperpath}
\markoddbranchafterxor{9/6}{markmidupperpath}{->}
\markevenbranchbeforetee{0,2,4,6,8,10,12,14}{markmidlowerpath}{<-}
\markoddbranch{1/0,3/4,5/12,7/8,9/6,11/2,13/10,15/14}{markmidlowerpath}{<-}
\markevenbranchaftertee{0/5,2/1,4/7,6/3,8/13,12/15,14/11}{markmidlowerpath}{<-}
\markcommonactivesboxes{}
}
\twineroundwokey{%s
\markevenbranches{6/3}{markmidupperpath}{->}
\markoddbranchbeforexor{}{markmidupperpath}
\markoddbranchafterxor{7/8}{markmidupperpath}{->}
\markevenbranchbeforetee{0,2,4,6,8,10,12,14}{markmidlowerpath}{<-}
\markoddbranch{1/0,3/4,5/12,7/8,11/2,13/10,15/14}{markmidlowerpath}{<-}
\markevenbranchaftertee{0/5,2/1,8/13,12/15,14/11}{markmidlowerpath}{<-}
\markcommonactivesboxes{3}
}
\twineroundwokey{%s
\markevenbranches{8/13}{markmidupperpath}{->}
\markoddbranchbeforexor{3}{markmidupperpath}
\markoddbranchafterxor{3/4,9/6}{markmidupperpath}{->}
\markevenbranchbeforetee{0,2,4,8,10,12,14}{markmidlowerpath}{<-}
\markoddbranch{1/0,5/12,11/2,13/10,15/14}{markmidlowerpath}{<-}
\markevenbranchaftertee{2/1,8/13,14/11}{markmidlowerpath}{<-}
\markcommonactivesboxes{}
}
\twineroundwokey{%s
\markevenbranches{4/7,6/3}{markmidupperpath}{->}
\markoddbranchbeforexor{13}{markmidupperpath}
\markoddbranchafterxor{5/12,7/8,13/10}{markmidupperpath}{->}
\markevenbranchbeforetee{0,2,10,12,14}{markmidlowerpath}{<-}
\markoddbranch{1/0,11/2,13/10}{markmidlowerpath}{<-}
\markevenbranchaftertee{2/1,14/11}{markmidlowerpath}{<-}
\markcommonactivesboxes{}
}
\twineroundwokey{%s
\markevenbranches{8/13,10/9,12/15}{markmidupperpath}{->}
\markoddbranchbeforexor{3,7}{markmidupperpath}
\markoddbranchafterxor{3/4,7/8,9/6,11/2,13/10}{markmidupperpath}{->}
\markevenbranchbeforetee{0,2,10}{markmidlowerpath}{<-}
\markoddbranch{1/0,11/2}{markmidlowerpath}{<-}
\markevenbranchaftertee{2/1}{markmidlowerpath}{<-}
\markcommonactivesboxes{5}
}
\twineroundwokey{%s
\markevenbranches{2/1,4/7,6/3,8/13,10/9}{markmidupperpath}{->}
\markoddbranchbeforexor{9,13,15}{markmidupperpath}
\markoddbranchafterxor{3/4,5/12,7/8,9/6,11/2,13/10,15/14}{markmidupperpath}{->}
\markevenbranchbeforetee{0,2}{markmidlowerpath}{<-}
\markoddbranch{1/0}{markmidlowerpath}{<-}
\markevenbranchaftertee{2/1}{markmidlowerpath}{<-}
\markcommonactivesboxes{}
}
\twineroundwokey{%s
\markevenbranches{2/1,4/7,6/3,8/13,10/9,12/15,14/11}{markmidupperpath}{->}
\markoddbranchbeforexor{1,3,7,9,13}{markmidupperpath}
\markoddbranchafterxor{1/0,3/4,5/12,7/8,9/6,11/2,13/10,15/14}{markmidupperpath}{->}
\markevenbranchbeforetee{0}{markmidlowerpath}{<-}
\markoddbranch{1/0}{markmidlowerpath}{<-}
\markevenbranchaftertee{}{markmidlowerpath}{<-}
\markcommonactivesboxes{}
}
\twineroundwokey{%s
\markevenbranches{0/5,2/1,4/7,6/3,8/13,10/9,12/15,14/11}{markmidupperpath}{->}
\markoddbranchbeforexor{1,3,7,9,11,13,15}{markmidupperpath}
\markoddbranchafterxor{1/0,3/4,5/12,7/8,9/6,11/2,13/10,15/14}{markmidupperpath}{->}
\markevenbranchbeforetee{0}{markmidlowerpath}{<-}
\markoddbranch{}{markmidlowerpath}{<-}
\markevenbranchaftertee{0/5}{markmidlowerpath}{<-}
\markcommonactivesboxes{}
}
\twineroundwokey{%s
\markuppercrossingdifferences{0,1,2,3,4,5,6,7,8,9,10,11,12,13,14,15}
\markevenbranchbeforetee{}{marklowerpath}{<-}
\markoddbranch{5/12}{marklowerpath}{<-}
\markevenbranchaftertee{4/7}{marklowerpath}{<-}
}
\twineroundwokey{%s
\markevenbranchbeforetee{12}{marklowerpath}{<-}
\markoddbranch{7/8}{marklowerpath}{<-}
\markevenbranchaftertee{6/3,12/15}{marklowerpath}{<-}
\markoutputdiff{3,8,15}
}
\begin{tikzpicture}[twinefig]
\foreach \z[evaluate=\z as \zf using int(4*\z)] in {0,...,15} {
\draw[gray] (\z,-3pt) node[below] {\tiny\zf};
\foreach \zb in {0,...,3} { \draw[gray] (\z+.25*\zb,0) -- +(0,-3pt); }
}
\foreach \z in {12,13,14,15,32,33,34,35,60,61,62,63} { \fill[tugblue] (.25*\z,-.75) circle[radius=3pt]; }
\end{tikzpicture}
\begin{comment}
#######################################################
Summary of the results:
A differential trail for EU:
Rounds	x                 pr     
--------------------------------
0	00000000a7009807  -4                
1	0000000000000a79  -2                
2	0000000000a70000  -2                
3	000000000a000000  none              
Weight: -8.00
-------------------------------------------------------
Sandwich 8 rounds in the middle with 2 active S-boxes
-------------------------------------------------------
A linear trail for EL:
Rounds	x                 pr     
--------------------------------
0	00000a0000000000  -2                
1	0000000c0000a000  -2                
2	00020000c000000a  none              
Weight: -4.00
#######################################################
differential effect of the upper trail: 2^(-8.00)
squared correlation of the lower trail: 2^(-4.00)
#######################################################

Total correlation = p*r*q^2 = 2^(-8.00) x r x 2^(-4.00)
2^(-15.00) <= Total correlation <= 2^(-13.00)
To compute the accurate value of total probability, r should be evaluated experimentally or using the DLCT framework

Number of attacked rounds: 13
Configuration: ru=3, rm=8, rl=2, rmu=0, rml=0, wu=1, wm=1, wl=1
\end{comment}
\caption{Differential-linear distinguisher for 13 rounds of \texttt{TWINE}.}
\label{fig:difflin_distinguisher}
\end{figure}
\end{document}
